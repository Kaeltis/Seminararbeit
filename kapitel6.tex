\chapter{Zusammenfassung}
Zusammenfassend lässt sich sagen, dass es zum einen sehr viele unterschiedliche Quellen für Projekte gibt, von welchen die Herstellerseiten oft die Projekte bieten, welche sich am Einfachsten mit dem entsprechenden 3D-Drucker umsetzen lassen, ohne viel Aufwand bei dem Druck oder der Nachbearbeitung zu erzeugen.

Auch bei den Materialien gibt es sehr viele aus welchen sich das Beste für den gewünschten Einsatzzweck auswählen lässt, hierbei lassen sich ebenfalls die Materialien der Hersteller von 3D-Druckern empfehlen, da die Standardeinstellungen der Slicer oft speziell für diese Entwickelt wurden.

Der größte Punkt, welcher die Umsetzung eines Projektes vereinfachen oder verkomplizieren kann, ist die Qualität des 3D-Druckers. In der Praxis zeigen sich deutliche Unterschiede zwischen Profi- und Hobbydruckern.

Die Profigeräte erzielen bereits mit den Standardeinstellungen sehr gute Ergebnisse, bei welchen die Nachbearbeitung einfach oder gar nicht nötig ist. Die Qualität der Drucke ist durch die gleichbleibende Materialqualität gewährleistet, Probleme treten kaum auf. Projekte lassen sich in kurzer Zeit einfach umsetzen.

Bei Hobbygeräten ist oft Bastelarbeit und viel Feingefühl notwendig. Die Nachbearbeitung ist meist aufwendig und durch fehlerhafte Drucke wird viel Material und Zeit verschwendet. Auch die Druckqualität schwankt aufgrund unklarer Herkunft und genauer Zusammensetzung der Kunststoffe. Projekte sind oft mit Hürden verbunden und der benötigte Zeitrahmen für die Umsetzung nur schwer einschätzbar.